% Prefacio

\chapter*{Prefacio} % Main chapter title

\label{Prefacio} % For referencing the chapter elsewhere, use \ref{Chapter1} 

\lhead{Prefacio} % This is for the header on each page - perhaps a shortened title

%----------------------------------------------------------------------------------------
A comienzos de la década de 1990, siendo todavía yo un niño, me escondía en la logia de la casa de mis padres, que estaba ubicada en la comuna de Ñuñoa, a desarmar cuanto artefacto cayera en mis manos. Según recuerdo, sumado a lo que mi madre me ha contado, una vez desarmé el único teléfono que había en la casa. Era uno de esos que ya no se ven, que hacen uso de un dial para marcar el número. Tal era mi curiosidad sobre esta máquina, que la desarmé completamente, solo para tratar de entender como se producía ese sonido sin igual que llamaba la atención de toda persona que estuviera en la casa. Por supuesto mi madre me retó porque no pude volver a armarlo, pero en secreto siempre supe que ella y mi padre se divertían con mis pequeñas aventuras. En esta misma casa, mi padre intentó montar una empresa de consultores con unos amigos, lo que significaba que al lado de mi pequeño taller clandestino tenía a mi disposición tres o cuatro computadores con sus flamantes pantallas Hércules frente a la cuales pasé tardes completas jugando los primeros vídeo juegos, y sin darme cuenta fue así como di mis primeros pasos en la computación. Algo más grande, cuando tenía 13 años, por el trabajo de mi padre nos fuimos a vivir a Concepción. Una ciudad bien particular, ya que el invierno da mucho tiempo para estar en el hogar. Ahí volví a encontrarme con la computación. Recuerdo que mientras mis compañeros estaban en clases, yo me escabullía para ir directo a las salas de computación, donde el encargado me permitía ayudarle a instalar Windows 95 en los computadores. Disquete tras disquete instalábamos el programa en cada computador hasta lograr que el famoso logo de la ventana saliera en la pantalla. Desde esa  época es que familiares y amigos me han pedido ayuda con sus máquinas. En la enseñanza media volvimos como familia a Santiago, volví a mi colegio de infancia, el Liceo San Agustín. Por ese tiempo junto a dos compañeros nos inscribimos en el concurso escolar de robótica de la Universidad Diego Portales. Aquí alumnos de la mencionada universidad nos enseñaron como se podía construir y programar un robot. No fue fácil y luego de un par de meses entre tantos manuales y simbología alienígena (diagramas eléctricos) tomamos la decisión de retirarnos. En el año 2005 ingresé a la Universidad Técnica Federico Santa María (USM). Llegué el 2006 a Valparaíso y lo primero que vi en el patio central fue un pequeño cartel que decía: "Taller del Centro de Robótica", junto a otro amigo que hice apenas ingresé a la USM, entusiasmados con esta idea de hacer nuestros propios robots, no lo dudamos y nos inscribimos para ser parte de ese taller. Una vez dentro del Centro de Robótica, conocí mucha gente con intereses similares a los míos, con gran conocimiento y más que nada, una gran solidaridad para compartir éste. Tardes completas dedicadas a aprender las oscuras artes de la electrónica, luchando con la frustración de armar un circuito y que éste no funcionara a la primera, fui paso a paso avanzando hasta poder construir mis primeros robots. Argo fue el primer proyecto de robótica en que me incluí, donde se trataba de construir un robot cuadrúpedo capaz de moverse en todas las direcciones en un plano utilizando la menor cantidad posible de grados de libertad. Luego vino LALO, proyecto que se destacó por ser de los primeros robots en hacer uso de un Smartphone como cámara y controlador de motores. En la actualidad me interesa más poder transmitir los conocimientos que tengo para introducir a la gente en las nuevas tecnologías existentes que están al alcance de todos.

Hacer Robots en Valparaíso no es una tarea sencilla, la falta de lugares especializados para comprar dificulta contar con los componentes necesarios para construir una máquina. Aunque no es imposible, a continuación explico como se hizo.



%----------------------------------------------------------------------------------------